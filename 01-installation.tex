% Options for packages loaded elsewhere
\PassOptionsToPackage{unicode}{hyperref}
\PassOptionsToPackage{hyphens}{url}
\PassOptionsToPackage{dvipsnames,svgnames,x11names}{xcolor}
%
\documentclass[
  letterpaper,
  DIV=11,
  numbers=noendperiod]{scrartcl}

\usepackage{amsmath,amssymb}
\usepackage{iftex}
\ifPDFTeX
  \usepackage[T1]{fontenc}
  \usepackage[utf8]{inputenc}
  \usepackage{textcomp} % provide euro and other symbols
\else % if luatex or xetex
  \usepackage{unicode-math}
  \defaultfontfeatures{Scale=MatchLowercase}
  \defaultfontfeatures[\rmfamily]{Ligatures=TeX,Scale=1}
\fi
\usepackage{lmodern}
\ifPDFTeX\else  
    % xetex/luatex font selection
\fi
% Use upquote if available, for straight quotes in verbatim environments
\IfFileExists{upquote.sty}{\usepackage{upquote}}{}
\IfFileExists{microtype.sty}{% use microtype if available
  \usepackage[]{microtype}
  \UseMicrotypeSet[protrusion]{basicmath} % disable protrusion for tt fonts
}{}
\makeatletter
\@ifundefined{KOMAClassName}{% if non-KOMA class
  \IfFileExists{parskip.sty}{%
    \usepackage{parskip}
  }{% else
    \setlength{\parindent}{0pt}
    \setlength{\parskip}{6pt plus 2pt minus 1pt}}
}{% if KOMA class
  \KOMAoptions{parskip=half}}
\makeatother
\usepackage{xcolor}
\setlength{\emergencystretch}{3em} % prevent overfull lines
\setcounter{secnumdepth}{-\maxdimen} % remove section numbering
% Make \paragraph and \subparagraph free-standing
\ifx\paragraph\undefined\else
  \let\oldparagraph\paragraph
  \renewcommand{\paragraph}[1]{\oldparagraph{#1}\mbox{}}
\fi
\ifx\subparagraph\undefined\else
  \let\oldsubparagraph\subparagraph
  \renewcommand{\subparagraph}[1]{\oldsubparagraph{#1}\mbox{}}
\fi


\providecommand{\tightlist}{%
  \setlength{\itemsep}{0pt}\setlength{\parskip}{0pt}}\usepackage{longtable,booktabs,array}
\usepackage{calc} % for calculating minipage widths
% Correct order of tables after \paragraph or \subparagraph
\usepackage{etoolbox}
\makeatletter
\patchcmd\longtable{\par}{\if@noskipsec\mbox{}\fi\par}{}{}
\makeatother
% Allow footnotes in longtable head/foot
\IfFileExists{footnotehyper.sty}{\usepackage{footnotehyper}}{\usepackage{footnote}}
\makesavenoteenv{longtable}
\usepackage{graphicx}
\makeatletter
\def\maxwidth{\ifdim\Gin@nat@width>\linewidth\linewidth\else\Gin@nat@width\fi}
\def\maxheight{\ifdim\Gin@nat@height>\textheight\textheight\else\Gin@nat@height\fi}
\makeatother
% Scale images if necessary, so that they will not overflow the page
% margins by default, and it is still possible to overwrite the defaults
% using explicit options in \includegraphics[width, height, ...]{}
\setkeys{Gin}{width=\maxwidth,height=\maxheight,keepaspectratio}
% Set default figure placement to htbp
\makeatletter
\def\fps@figure{htbp}
\makeatother

\KOMAoption{captions}{tableheading}
\makeatletter
\makeatother
\makeatletter
\makeatother
\makeatletter
\@ifpackageloaded{caption}{}{\usepackage{caption}}
\AtBeginDocument{%
\ifdefined\contentsname
  \renewcommand*\contentsname{Table of contents}
\else
  \newcommand\contentsname{Table of contents}
\fi
\ifdefined\listfigurename
  \renewcommand*\listfigurename{List of Figures}
\else
  \newcommand\listfigurename{List of Figures}
\fi
\ifdefined\listtablename
  \renewcommand*\listtablename{List of Tables}
\else
  \newcommand\listtablename{List of Tables}
\fi
\ifdefined\figurename
  \renewcommand*\figurename{Figure}
\else
  \newcommand\figurename{Figure}
\fi
\ifdefined\tablename
  \renewcommand*\tablename{Table}
\else
  \newcommand\tablename{Table}
\fi
}
\@ifpackageloaded{float}{}{\usepackage{float}}
\floatstyle{ruled}
\@ifundefined{c@chapter}{\newfloat{codelisting}{h}{lop}}{\newfloat{codelisting}{h}{lop}[chapter]}
\floatname{codelisting}{Listing}
\newcommand*\listoflistings{\listof{codelisting}{List of Listings}}
\makeatother
\makeatletter
\@ifpackageloaded{caption}{}{\usepackage{caption}}
\@ifpackageloaded{subcaption}{}{\usepackage{subcaption}}
\makeatother
\makeatletter
\@ifpackageloaded{tcolorbox}{}{\usepackage[skins,breakable]{tcolorbox}}
\makeatother
\makeatletter
\@ifundefined{shadecolor}{\definecolor{shadecolor}{rgb}{.97, .97, .97}}
\makeatother
\makeatletter
\makeatother
\makeatletter
\makeatother
\ifLuaTeX
  \usepackage{selnolig}  % disable illegal ligatures
\fi
\IfFileExists{bookmark.sty}{\usepackage{bookmark}}{\usepackage{hyperref}}
\IfFileExists{xurl.sty}{\usepackage{xurl}}{} % add URL line breaks if available
\urlstyle{same} % disable monospaced font for URLs
\hypersetup{
  pdftitle={1.1 Installation},
  colorlinks=true,
  linkcolor={blue},
  filecolor={Maroon},
  citecolor={Blue},
  urlcolor={Blue},
  pdfcreator={LaTeX via pandoc}}

\title{1.1 Installation}
\author{}
\date{}

\begin{document}
\maketitle
\ifdefined\Shaded\renewenvironment{Shaded}{\begin{tcolorbox}[sharp corners, borderline west={3pt}{0pt}{shadecolor}, boxrule=0pt, breakable, enhanced, interior hidden, frame hidden]}{\end{tcolorbox}}\fi

Was nötig ist, um mit R arbeiten zu können, hängt davon ab, welches
Betriebssystem auf eurem Gerät installiert ist.

\hypertarget{windows-und-macos}{%
\subsection{Windows und MacOS}\label{windows-und-macos}}

Auf Windows und MacOS müssen wir zwei Programme installieren: zum einen
die R-Distribution (das heißt, das Paket, mit dem man R-Code ausführen
kann), zum anderen eine Entwicklungsumgebung, in der wir unseren R-Code
schreiben können und die uns zahlreiche Hilfsmittel zum Schreiben
unseres Codes und zum Anzeigen unserer Ergebnisse bietet.

In den vergangenen Jahren habe ich einige verschiedene ausprobiert, aber
ich bin am Ende immer zu der bekanntesten Software für R zurückgekehrt,
nämlich \emph{RStudio}. RStudio ist eine von der Firma Posit entwickelte
grafische Oberfläche, mit der ihr euren R-Code strukturiert schreiben
und die Ergebnisse eurer Analyse ansprechend aufbereiten könnt.
Natürlich könnt ihr auch andere Programme ausprobieren und benutzen,
aber die weiteren Erklärungen sind auf RStudio ausgerichtet.

\begin{enumerate}
\def\labelenumi{\arabic{enumi}.}
\item
  \textbf{R-Distribution}

  R ist eine frei verfügbare Software und kann somit von jedem kostenlos
  heruntergeladen werden.

  Ihr könnt sie installieren, indem ihr die Seite
  \href{https://cran.r-project.org}{cran.r-project.org} aufruft und dort
  auf den Link \emph{Download R for MacOS} bzw. \emph{Download R for
  Windows} klickt. Wenn ihr ein MacBook habt, wählt ihr auf dieser Seite
  die passende Version für euer Gerät aus. Bei Windows klicken wir auf
  \emph{base} und dann auf den Downloadlink ganz oben.

  Folgt dann den Installationsanweisungen, dabei könnt ihr alles bei den
  Standardeinstellungen belassen.
\item
  \textbf{Entwicklungsumgebung}

  Damit wir die R-Distribution sinnvoll nutzen können, benötigen wir
  eine Entwicklungsumgebung, in unserem Fall RStudio.

  Um RStudio zu installieren, geht auf die Seite
  \href{https://posit.co/download/rstudio-desktop}{posit.co/download/rstudio-desktop}.
  Dort seht ihr den Punkt \emph{2:} \emph{Install RStudio}. Es sollte
  bereits das richtige Betriebssystem vorgewählt sein, ansonsten finden
  sich darunter noch weitere Installer. Ladet nun durch Klick auf den
  Button \emph{Download RStudio for \ldots{}} das Installationsprogramm
  herunter und folgt den Installationsanweisungen.
\end{enumerate}

Nun sollten sowohl R als auch RStudio auf eurem Gerät installiert sein.

\hypertarget{ipads-und-android-tablets}{%
\subsection{iPads und Android-Tablets}\label{ipads-und-android-tablets}}

Auch auf einem iPad oder einem Android-Tablet kann man mit R
programmieren, auch wenn das ohne Tastatur und Maus wahrscheinlich
schwierig ist. Sollte es so sein, dass ihr keinen Zugang zu einem Laptop
habt, ist dies jedoch eine mögliche Lösung:

Wir installieren kein Programm auf unserem Gerät. Stattdessen erstellen
wir ein Benutzerkonto bei Posit, dem Anbieter von RStudio. Auf diese
Weise können wir, wenn auch nur mit beschränkter Rechenzeit, in unserem
Browser programmieren. Hierzu geht ihr auf die Seite von Posit,
\href{https://posit.cloud}{posit.cloud}, und klickt dort auf \emph{Get
started}. Wählt den kostenlosen Plan (Cloud Free) aus und registriert
euch dann unter \emph{Sign up}. Wenn das geklappt hat, solltet ihr
bereit sein, um in R zu programmieren.

\hypertarget{linux}{%
\subsection{Linux}\label{linux}}

Solltet ihr eine Linux-Distribution auf eurem Laptop installiert haben,
dann findet ihr auf YouTube sowie mit einer kurzen Internetrecherche
zahlreiche Anleitungen, wie ihr R auf eurer Plattform installieren
könnt. Ich gehe nicht näher auf diesen Fall ein, da man je nach
Distribution unterschiedlich verfahren muss und ich dem geneigten
Linux-Nutzer unterstelle, selbst in der Lage zu sein, die Installation
durchzuführen.



\end{document}
